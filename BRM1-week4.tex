% Options for packages loaded elsewhere
\PassOptionsToPackage{unicode}{hyperref}
\PassOptionsToPackage{hyphens}{url}
%
\documentclass[
]{article}
\usepackage{lmodern}
\usepackage{amssymb,amsmath}
\usepackage{ifxetex,ifluatex}
\ifnum 0\ifxetex 1\fi\ifluatex 1\fi=0 % if pdftex
  \usepackage[T1]{fontenc}
  \usepackage[utf8]{inputenc}
  \usepackage{textcomp} % provide euro and other symbols
\else % if luatex or xetex
  \usepackage{unicode-math}
  \defaultfontfeatures{Scale=MatchLowercase}
  \defaultfontfeatures[\rmfamily]{Ligatures=TeX,Scale=1}
\fi
% Use upquote if available, for straight quotes in verbatim environments
\IfFileExists{upquote.sty}{\usepackage{upquote}}{}
\IfFileExists{microtype.sty}{% use microtype if available
  \usepackage[]{microtype}
  \UseMicrotypeSet[protrusion]{basicmath} % disable protrusion for tt fonts
}{}
\makeatletter
\@ifundefined{KOMAClassName}{% if non-KOMA class
  \IfFileExists{parskip.sty}{%
    \usepackage{parskip}
  }{% else
    \setlength{\parindent}{0pt}
    \setlength{\parskip}{6pt plus 2pt minus 1pt}}
}{% if KOMA class
  \KOMAoptions{parskip=half}}
\makeatother
\usepackage{xcolor}
\IfFileExists{xurl.sty}{\usepackage{xurl}}{} % add URL line breaks if available
\IfFileExists{bookmark.sty}{\usepackage{bookmark}}{\usepackage{hyperref}}
\hypersetup{
  pdftitle={BRM1\_IBA\_2021: Week 4},
  pdfauthor={Demi Maric, Ashley, Merle, Team 117},
  hidelinks,
  pdfcreator={LaTeX via pandoc}}
\urlstyle{same} % disable monospaced font for URLs
\usepackage[margin=1in]{geometry}
\usepackage{color}
\usepackage{fancyvrb}
\newcommand{\VerbBar}{|}
\newcommand{\VERB}{\Verb[commandchars=\\\{\}]}
\DefineVerbatimEnvironment{Highlighting}{Verbatim}{commandchars=\\\{\}}
% Add ',fontsize=\small' for more characters per line
\usepackage{framed}
\definecolor{shadecolor}{RGB}{248,248,248}
\newenvironment{Shaded}{\begin{snugshade}}{\end{snugshade}}
\newcommand{\AlertTok}[1]{\textcolor[rgb]{0.94,0.16,0.16}{#1}}
\newcommand{\AnnotationTok}[1]{\textcolor[rgb]{0.56,0.35,0.01}{\textbf{\textit{#1}}}}
\newcommand{\AttributeTok}[1]{\textcolor[rgb]{0.77,0.63,0.00}{#1}}
\newcommand{\BaseNTok}[1]{\textcolor[rgb]{0.00,0.00,0.81}{#1}}
\newcommand{\BuiltInTok}[1]{#1}
\newcommand{\CharTok}[1]{\textcolor[rgb]{0.31,0.60,0.02}{#1}}
\newcommand{\CommentTok}[1]{\textcolor[rgb]{0.56,0.35,0.01}{\textit{#1}}}
\newcommand{\CommentVarTok}[1]{\textcolor[rgb]{0.56,0.35,0.01}{\textbf{\textit{#1}}}}
\newcommand{\ConstantTok}[1]{\textcolor[rgb]{0.00,0.00,0.00}{#1}}
\newcommand{\ControlFlowTok}[1]{\textcolor[rgb]{0.13,0.29,0.53}{\textbf{#1}}}
\newcommand{\DataTypeTok}[1]{\textcolor[rgb]{0.13,0.29,0.53}{#1}}
\newcommand{\DecValTok}[1]{\textcolor[rgb]{0.00,0.00,0.81}{#1}}
\newcommand{\DocumentationTok}[1]{\textcolor[rgb]{0.56,0.35,0.01}{\textbf{\textit{#1}}}}
\newcommand{\ErrorTok}[1]{\textcolor[rgb]{0.64,0.00,0.00}{\textbf{#1}}}
\newcommand{\ExtensionTok}[1]{#1}
\newcommand{\FloatTok}[1]{\textcolor[rgb]{0.00,0.00,0.81}{#1}}
\newcommand{\FunctionTok}[1]{\textcolor[rgb]{0.00,0.00,0.00}{#1}}
\newcommand{\ImportTok}[1]{#1}
\newcommand{\InformationTok}[1]{\textcolor[rgb]{0.56,0.35,0.01}{\textbf{\textit{#1}}}}
\newcommand{\KeywordTok}[1]{\textcolor[rgb]{0.13,0.29,0.53}{\textbf{#1}}}
\newcommand{\NormalTok}[1]{#1}
\newcommand{\OperatorTok}[1]{\textcolor[rgb]{0.81,0.36,0.00}{\textbf{#1}}}
\newcommand{\OtherTok}[1]{\textcolor[rgb]{0.56,0.35,0.01}{#1}}
\newcommand{\PreprocessorTok}[1]{\textcolor[rgb]{0.56,0.35,0.01}{\textit{#1}}}
\newcommand{\RegionMarkerTok}[1]{#1}
\newcommand{\SpecialCharTok}[1]{\textcolor[rgb]{0.00,0.00,0.00}{#1}}
\newcommand{\SpecialStringTok}[1]{\textcolor[rgb]{0.31,0.60,0.02}{#1}}
\newcommand{\StringTok}[1]{\textcolor[rgb]{0.31,0.60,0.02}{#1}}
\newcommand{\VariableTok}[1]{\textcolor[rgb]{0.00,0.00,0.00}{#1}}
\newcommand{\VerbatimStringTok}[1]{\textcolor[rgb]{0.31,0.60,0.02}{#1}}
\newcommand{\WarningTok}[1]{\textcolor[rgb]{0.56,0.35,0.01}{\textbf{\textit{#1}}}}
\usepackage{graphicx,grffile}
\makeatletter
\def\maxwidth{\ifdim\Gin@nat@width>\linewidth\linewidth\else\Gin@nat@width\fi}
\def\maxheight{\ifdim\Gin@nat@height>\textheight\textheight\else\Gin@nat@height\fi}
\makeatother
% Scale images if necessary, so that they will not overflow the page
% margins by default, and it is still possible to overwrite the defaults
% using explicit options in \includegraphics[width, height, ...]{}
\setkeys{Gin}{width=\maxwidth,height=\maxheight,keepaspectratio}
% Set default figure placement to htbp
\makeatletter
\def\fps@figure{htbp}
\makeatother
\setlength{\emergencystretch}{3em} % prevent overfull lines
\providecommand{\tightlist}{%
  \setlength{\itemsep}{0pt}\setlength{\parskip}{0pt}}
\setcounter{secnumdepth}{-\maxdimen} % remove section numbering

\title{BRM1\_IBA\_2021: Week 4}
\author{Demi Maric, Ashley, Merle, Team 117}
\date{11/3/2020}

\begin{document}
\maketitle

\hypertarget{before-you-start-install-your-working-environment}{%
\subsection{Before you start, install your working
environment}\label{before-you-start-install-your-working-environment}}

\begin{itemize}
\tightlist
\item
  check your working directory
\item
  setwd() to change working directory
\item
  remove all objects, start with an empty environment
\end{itemize}

\begin{Shaded}
\begin{Highlighting}[]
\KeywordTok{getwd}\NormalTok{()}
\end{Highlighting}
\end{Shaded}

\begin{verbatim}
## [1] "C:/Users/Laptop/Documents/github/BRM117"
\end{verbatim}

\begin{Shaded}
\begin{Highlighting}[]
\KeywordTok{rm}\NormalTok{(}\DataTypeTok{list=}\KeywordTok{ls}\NormalTok{()) }
\end{Highlighting}
\end{Shaded}

\hypertarget{assignment-week-4}{%
\section{Assignment week 4}\label{assignment-week-4}}

Use the IMDB data file that you will analyze during the coming two
weeks, the description of the variables, and the excel sheets that you
need to hand in. You can find the files in the zip file you downloaded
in week 1.

Read the IMDB data file into R: - download data file in your working
directory - read data file
\texttt{sep\ =\ \textquotesingle{}\textquotesingle{}} - save data in an
R object - inspect dataframe

\begin{Shaded}
\begin{Highlighting}[]
\KeywordTok{setwd}\NormalTok{(}\StringTok{"~/github/BRM117"}\NormalTok{)}
\NormalTok{df<-}\KeywordTok{read.csv}\NormalTok{(}\StringTok{"Netherlands_USA.csv"}\NormalTok{, }\DataTypeTok{sep=}\StringTok{","}\NormalTok{)}
\KeywordTok{str}\NormalTok{(df)}
\end{Highlighting}
\end{Shaded}

\begin{verbatim}
## 'data.frame':    1153 obs. of  18 variables:
##  $ Movie_ID                            : int  1 1 2 2 3 3 4 4 5 5 ...
##  $ Movie                               : chr  "42" "42" "2012" "2012" ...
##  $ Release_date                        : chr  "4/12/2013" "4/12/2013" "11/13/2009" "11/13/2009" ...
##  $ Distributor                         : chr  "WB" "WB" "Sony" "Sony" ...
##  $ MPAA_rating                         : chr  "PG-13" "PG-13" "PG-13" "PG-13" ...
##  $ Runtime                             : int  128 128 158 158 95 95 102 102 109 109 ...
##  $ Production_budget                   : num  40 40 200 200 7.5 7.5 NA NA 61 61 ...
##  $ Genre1                              : chr  "Biography" "Biography" "Action" "Action" ...
##  $ Genre2                              : chr  "Drama" "Drama" "Adventure" "Adventure" ...
##  $ Genre3                              : chr  "Sport" "Sport" "Sci-Fi" "Sci-Fi" ...
##  $ imdb.com_rating                     : num  7.5 7.5 5.8 5.8 7.8 7.8 6.4 6.4 6.8 6.8 ...
##  $ Country                             : chr  "Netherlands" "USA" "Netherlands" "USA" ...
##  $ Opening_revenues                    : int  NA 27487144 1880951 65237614 60289 834501 307757 23722310 501591 27059130 ...
##  $ Opening_screens                     : int  NA 3003 127 3404 15 27 70 3255 75 3025 ...
##  $ Opening_revenues_indexed            : num  NA 0.1578 0.527 0.4567 0.0169 ...
##  $ Opening_screens_indexed             : num  NA 0.7061 0.5292 0.7871 0.0625 ...
##  $ Opening_revenues_indexed_transformed: num  NA -1.674 0.108 -0.174 -4.064 ...
##  $ Opening_screens_indexed_transformed : num  NA 0.876 0.117 1.307 -2.708 ...
\end{verbatim}

\hypertarget{dummy-variable}{%
\section{1. Dummy variable}\label{dummy-variable}}

Each movie has a genre attached to it. Some even three. Inspect the
first genre \texttt{Genre1}, and set all empty strings
\texttt{df\$Genre1=="\ "} (mind the space in between the quotation
marks!) to \texttt{NA}. Create a dummy variable equal to 1 if a movie's
genre is \texttt{Action} and 0 if it has another genre. First inspect
the variable for frequencies, levels and missings! Missing values in
\texttt{Genre1} should also be missing in the dummy variable. Report the
frequencies.

\begin{Shaded}
\begin{Highlighting}[]
\CommentTok{# write your R code here}
\KeywordTok{table}\NormalTok{(df}\OperatorTok{$}\NormalTok{Genre1)}
\end{Highlighting}
\end{Shaded}

\begin{verbatim}
## 
##                  Action   Adventure   Animation   Biography      Comedy 
##           2         388          89          24          53         271 
##       Crime Documentary       Drama     Fantasy      Horror     Mystery 
##          50          24         150           5          79           6 
##     Romance      Sci-Fi       Short 
##           2           4           6
\end{verbatim}

\begin{Shaded}
\begin{Highlighting}[]
\NormalTok{df}\OperatorTok{$}\NormalTok{Genre1}\FloatTok{.1}\NormalTok{<-}\OtherTok{NA}
\NormalTok{df}\OperatorTok{$}\NormalTok{Genre1}\FloatTok{.1}\NormalTok{[df}\OperatorTok{$}\NormalTok{Genre1}\OperatorTok{==}\StringTok{" "}\NormalTok{]<-}\OtherTok{NA}
\NormalTok{df}\OperatorTok{$}\NormalTok{Genre1}\FloatTok{.1}\NormalTok{[df}\OperatorTok{$}\NormalTok{Genre1}\OperatorTok{==}\StringTok{"Action"}\NormalTok{]<-}\DecValTok{1}
\NormalTok{df}\OperatorTok{$}\NormalTok{Genre1}\FloatTok{.1}\NormalTok{<-}\KeywordTok{ifelse}\NormalTok{(df}\OperatorTok{$}\NormalTok{Genre1 }\OperatorTok{==}\StringTok{ "Action"}\NormalTok{, }\DecValTok{1}\NormalTok{, }\DecValTok{0}\NormalTok{)}
\KeywordTok{tail}\NormalTok{(df}\OperatorTok{$}\NormalTok{Genre1}\FloatTok{.1}\NormalTok{)}
\end{Highlighting}
\end{Shaded}

\begin{verbatim}
## [1] 0 0 0 0 0 0
\end{verbatim}

\begin{Shaded}
\begin{Highlighting}[]
\KeywordTok{table}\NormalTok{(df}\OperatorTok{$}\NormalTok{Genre1, df}\OperatorTok{$}\NormalTok{Genre1}\FloatTok{.1}\NormalTok{)}
\end{Highlighting}
\end{Shaded}

\begin{verbatim}
##              
##                 0   1
##                 2   0
##   Action        0 388
##   Adventure    89   0
##   Animation    24   0
##   Biography    53   0
##   Comedy      271   0
##   Crime        50   0
##   Documentary  24   0
##   Drama       150   0
##   Fantasy       5   0
##   Horror       79   0
##   Mystery       6   0
##   Romance       2   0
##   Sci-Fi        4   0
##   Short         6   0
\end{verbatim}

\begin{Shaded}
\begin{Highlighting}[]
\KeywordTok{sum}\NormalTok{(}\KeywordTok{is.na}\NormalTok{(df}\OperatorTok{$}\NormalTok{Genre1))}
\end{Highlighting}
\end{Shaded}

\begin{verbatim}
## [1] 0
\end{verbatim}

\begin{Shaded}
\begin{Highlighting}[]
\KeywordTok{sum}\NormalTok{(}\KeywordTok{is.na}\NormalTok{(df}\OperatorTok{$}\NormalTok{Genre1}\FloatTok{.1}\NormalTok{))}
\end{Highlighting}
\end{Shaded}

\begin{verbatim}
## [1] 0
\end{verbatim}

\begin{Shaded}
\begin{Highlighting}[]
\KeywordTok{length}\NormalTok{(df}\OperatorTok{$}\NormalTok{Genre1}\FloatTok{.1}\NormalTok{[df}\OperatorTok{$}\NormalTok{Genre1}\FloatTok{.1}\OperatorTok{==}\DecValTok{1}\NormalTok{])}
\end{Highlighting}
\end{Shaded}

\begin{verbatim}
## [1] 388
\end{verbatim}

\begin{Shaded}
\begin{Highlighting}[]
\KeywordTok{length}\NormalTok{(df}\OperatorTok{$}\NormalTok{Genre1}\FloatTok{.1}\NormalTok{[df}\OperatorTok{$}\NormalTok{Genre1}\FloatTok{.1}\OperatorTok{==}\DecValTok{0}\NormalTok{])}
\end{Highlighting}
\end{Shaded}

\begin{verbatim}
## [1] 765
\end{verbatim}

\hypertarget{inspect-variables}{%
\section{2. Inspect variables}\label{inspect-variables}}

First, set all empty strings in MPAA rating to \texttt{NA}. Then inspect
what the mean and standard deviation of IMDB-rating per MPAA-rating and
country. Use \texttt{tapply}.

\begin{Shaded}
\begin{Highlighting}[]
\CommentTok{# write your R code here}
\NormalTok{df}\OperatorTok{$}\NormalTok{MPAA_rating[df}\OperatorTok{$}\NormalTok{MPAA_rating}\OperatorTok{==}\StringTok{" "}\NormalTok{]<-}\OtherTok{NA}
\NormalTok{df}\OperatorTok{$}\NormalTok{MPAA_rating[df}\OperatorTok{$}\NormalTok{MPAA_rating}\OperatorTok{==}\StringTok{"NA"}\NormalTok{]<-}\OtherTok{NA}
\NormalTok{MPAACountry<-}\KeywordTok{interaction}\NormalTok{(df}\OperatorTok{$}\NormalTok{MPAA_rating,df}\OperatorTok{$}\NormalTok{Country)}
\KeywordTok{tapply}\NormalTok{(df}\OperatorTok{$}\NormalTok{imdb.com_rating,MPAACountry,sd,}\DataTypeTok{na.rm=}\OtherTok{TRUE}\NormalTok{)}
\end{Highlighting}
\end{Shaded}

\begin{verbatim}
##     G.Netherlands    PG.Netherlands PG-13.Netherlands     R.Netherlands 
##         2.1124630         1.0200854         0.9734573         0.8730892 
##             G.USA            PG.USA         PG-13.USA             R.USA 
##         2.1124630         0.9977753         0.9716069         0.8756012
\end{verbatim}

\begin{Shaded}
\begin{Highlighting}[]
\KeywordTok{tapply}\NormalTok{(df}\OperatorTok{$}\NormalTok{imdb.com_rating,MPAACountry,mean,}\DataTypeTok{na.rm=}\OtherTok{TRUE}\NormalTok{)}
\end{Highlighting}
\end{Shaded}

\begin{verbatim}
##     G.Netherlands    PG.Netherlands PG-13.Netherlands     R.Netherlands 
##          5.933333          5.909091          6.411896          6.470707 
##             G.USA            PG.USA         PG-13.USA             R.USA 
##          5.933333          5.900000          6.407246          6.454976
\end{verbatim}

\hypertarget{simple-regression}{%
\section{3. Simple regression}\label{simple-regression}}

Perform a simple regression using the MPAA\_rating to explain opening
revenues. Note that MPAA\_rating is treated as a factor by R. What does
the intercept represent? Calculate the absolute mean difference between
PG and PG-13 rated movies using the beta coefficients. Can you tell
whether this difference is significant?

\begin{Shaded}
\begin{Highlighting}[]
\CommentTok{# write your R code here}
\KeywordTok{class}\NormalTok{(df}\OperatorTok{$}\NormalTok{Opening_revenues)}
\end{Highlighting}
\end{Shaded}

\begin{verbatim}
## [1] "integer"
\end{verbatim}

\begin{Shaded}
\begin{Highlighting}[]
\KeywordTok{class}\NormalTok{(df}\OperatorTok{$}\NormalTok{MPAA_rating)}
\end{Highlighting}
\end{Shaded}

\begin{verbatim}
## [1] "character"
\end{verbatim}

\begin{Shaded}
\begin{Highlighting}[]
\NormalTok{OR2<-}\KeywordTok{as.character}\NormalTok{(df}\OperatorTok{$}\NormalTok{Opening_revenues)}
\KeywordTok{summary}\NormalTok{(}\KeywordTok{lm}\NormalTok{(OR2}\OperatorTok{~}\NormalTok{df}\OperatorTok{$}\NormalTok{MPAA_rating,df))}
\end{Highlighting}
\end{Shaded}

\begin{verbatim}
## 
## Call:
## lm(formula = OR2 ~ df$MPAA_rating, data = df)
## 
## Residuals:
##       Min        1Q    Median        3Q       Max 
## -18056954 -12850789  -8871258   4565804 189364184 
## 
## Coefficients:
##                     Estimate Std. Error t value Pr(>|t|)
## (Intercept)          8933775    6338213   1.410    0.159
## df$MPAA_ratingPG     4090199    6663458   0.614    0.539
## df$MPAA_ratingPG-13  9140749    6427939   1.422    0.155
## df$MPAA_ratingR      1273879    6454826   0.197    0.844
## 
## Residual standard error: 23720000 on 1011 degrees of freedom
##   (138 observations deleted due to missingness)
## Multiple R-squared:  0.0239, Adjusted R-squared:  0.02101 
## F-statistic: 8.253 on 3 and 1011 DF,  p-value: 1.991e-05
\end{verbatim}

\begin{Shaded}
\begin{Highlighting}[]
\DecValTok{9140749-4090198} 
\end{Highlighting}
\end{Shaded}

\begin{verbatim}
## [1] 5050551
\end{verbatim}

\begin{Shaded}
\begin{Highlighting}[]
\KeywordTok{print}\NormalTok{(}\StringTok{"test"}\NormalTok{)}
\end{Highlighting}
\end{Shaded}

\begin{verbatim}
## [1] "test"
\end{verbatim}

\hypertarget{anova-vs-simple-regression}{%
\section{4. ANOVA vs simple
regression}\label{anova-vs-simple-regression}}

Compare the result you obtained in exercise 4 with an ANOVA test on the
original variable MPAA rating. To which mean all the means are compared
in a standard ANOVA test? Report the \(F\)-value. Inspect the mean
differences using a \texttt{TukeyHSD} test.

\begin{Shaded}
\begin{Highlighting}[]
\CommentTok{# write your R code here}
\NormalTok{aov1 <-}\StringTok{ }\KeywordTok{aov}\NormalTok{(df}\OperatorTok{$}\NormalTok{Opening_revenues }\OperatorTok{~}\StringTok{ }\NormalTok{df}\OperatorTok{$}\NormalTok{MPAA_rating)}
\KeywordTok{summary}\NormalTok{(aov1)}
\end{Highlighting}
\end{Shaded}

\begin{verbatim}
##                  Df    Sum Sq   Mean Sq F value   Pr(>F)    
## df$MPAA_rating    3 1.392e+16 4.641e+15   8.253 1.99e-05 ***
## Residuals      1011 5.686e+17 5.624e+14                     
## ---
## Signif. codes:  0 '***' 0.001 '**' 0.01 '*' 0.05 '.' 0.1 ' ' 1
## 138 observations deleted due to missingness
\end{verbatim}

\begin{Shaded}
\begin{Highlighting}[]
\KeywordTok{TukeyHSD}\NormalTok{(aov1)}
\end{Highlighting}
\end{Shaded}

\begin{verbatim}
##   Tukey multiple comparisons of means
##     95% family-wise confidence level
## 
## Fit: aov(formula = df$Opening_revenues ~ df$MPAA_rating)
## 
## $`df$MPAA_rating`
##              diff         lwr      upr     p adj
## PG-G      4090199 -13056938.2 21237335 0.9276958
## PG-13-G   9140749  -7400325.0 25681823 0.4857961
## R-G       1273879 -15336382.0 17884141 0.9972773
## PG-13-PG  5050551   -914972.2 11016073 0.1298749
## R-PG     -2816319  -8971082.7  3338444 0.6412286
## R-PG-13  -7866870 -12045858.7 -3687881 0.0000087
\end{verbatim}

\begin{Shaded}
\begin{Highlighting}[]
\CommentTok{## F-value: 8.253}
\end{Highlighting}
\end{Shaded}

What is the between variance and what is the within variance? Although
you can read this from the ANOVA output, you can also calculate these
numbers yourself. The formula for the between variance of groups j=
(1,2\ldots k):

\[between=\sum_{j=1}^kn_{j}\left(\overline{X}_{j}-\overline{X}\right)^2\]
where \(\overline{X}\) is the overall mean. Usually one reports the mean
square, where you divide by \(k-1\).

Calculate the between variance yourself. Make sure you remove the
missings before you start! Check with \texttt{aov} calculated above.
Show your code below:

\begin{Shaded}
\begin{Highlighting}[]
\CommentTok{# first use na.omit to delete all the missing values on opening revenues and MPAArating}
\NormalTok{dff<-}\StringTok{ }\KeywordTok{na.omit}\NormalTok{(df[,}\KeywordTok{c}\NormalTok{(}\StringTok{"Opening_revenues"}\NormalTok{, }\StringTok{"MPAA_rating"}\NormalTok{)])}
\CommentTok{# then get the variables and save them in objects to call later}
\NormalTok{y<-}\StringTok{ }\NormalTok{dff}\OperatorTok{$}\NormalTok{Opening_revenues}
\NormalTok{x<-}\StringTok{ }\NormalTok{dff}\OperatorTok{$}\NormalTok{MPAA_rating}
\NormalTok{groups <-}\StringTok{ }\KeywordTok{levels}\NormalTok{(}\KeywordTok{factor}\NormalTok{(dff}\OperatorTok{$}\NormalTok{MPAA_rating))}
\KeywordTok{unique}\NormalTok{(groups)}
\end{Highlighting}
\end{Shaded}

\begin{verbatim}
## [1] "G"     "PG"    "PG-13" "R"
\end{verbatim}

\begin{Shaded}
\begin{Highlighting}[]
\NormalTok{n1 <-}\StringTok{ }\KeywordTok{length}\NormalTok{(y[x}\OperatorTok{==}\StringTok{"G"}\NormalTok{])}
\NormalTok{n2<-}\StringTok{ }\KeywordTok{length}\NormalTok{(y[x}\OperatorTok{==}\StringTok{"PG"}\NormalTok{])}
\NormalTok{n3<-}\StringTok{ }\KeywordTok{length}\NormalTok{(y[x}\OperatorTok{==}\StringTok{"PG-13"}\NormalTok{])}
\NormalTok{n4<-}\StringTok{ }\KeywordTok{length}\NormalTok{(y[x}\OperatorTok{==}\StringTok{"R"}\NormalTok{])}
\ControlFlowTok{for}\NormalTok{ (group }\ControlFlowTok{in}\NormalTok{ groups) \{}
\NormalTok{ nj <-}\StringTok{ }\KeywordTok{length}\NormalTok{(y[x}\OperatorTok{==}\NormalTok{group]) }
\NormalTok{ E<-}\StringTok{ }\NormalTok{(}\KeywordTok{mean}\NormalTok{(y[x}\OperatorTok{==}\NormalTok{group])}\OperatorTok{-}\KeywordTok{mean}\NormalTok{(y))}\OperatorTok{^}\DecValTok{2}
\NormalTok{ between <-}\StringTok{ }\DecValTok{0}
\NormalTok{ between <-}\StringTok{ }\NormalTok{between }\OperatorTok{+}\StringTok{ }\NormalTok{nj}\OperatorTok{*}\NormalTok{E}
 
 \KeywordTok{print}\NormalTok{(between)}
\NormalTok{\}}
\end{Highlighting}
\end{Shaded}

\begin{verbatim}
## [1] 4.129244e+14
## [1] 2.390625e+14
## [1] 6.757648e+15
## [1] 6.514848e+15
\end{verbatim}

\begin{Shaded}
\begin{Highlighting}[]
\CommentTok{# create a score of between variance of zero}
\NormalTok{between}\OperatorTok{/}\NormalTok{(}\KeywordTok{length}\NormalTok{(groups)}\OperatorTok{-}\DecValTok{1}\NormalTok{)}
\end{Highlighting}
\end{Shaded}

\begin{verbatim}
## [1] 2.171616e+15
\end{verbatim}

\begin{Shaded}
\begin{Highlighting}[]
\CommentTok{# start for loop}
\CommentTok{# loop over the groups}
\CommentTok{# for each group you will calculate (1) the length which is nj, the group mean}
\CommentTok{# fill in the formula, and add each run of the loop}
\CommentTok{# the between variance of that group to the overall between variance}

\CommentTok{# divide by number of groups - 1 to get mean square of between variance}

\CommentTok{# check with aov}
\KeywordTok{summary}\NormalTok{(aov1)}
\end{Highlighting}
\end{Shaded}

\begin{verbatim}
##                  Df    Sum Sq   Mean Sq F value   Pr(>F)    
## df$MPAA_rating    3 1.392e+16 4.641e+15   8.253 1.99e-05 ***
## Residuals      1011 5.686e+17 5.624e+14                     
## ---
## Signif. codes:  0 '***' 0.001 '**' 0.01 '*' 0.05 '.' 0.1 ' ' 1
## 138 observations deleted due to missingness
\end{verbatim}

The formula for the within variance is as follows:

\[within=\sum_{j=1}^k \sum_{i=1}^n\left(X_{ij}-\overline{X}_{j}\right)^2\]

Calculate the within variance yourself.

\begin{Shaded}
\begin{Highlighting}[]
\CommentTok{# same as before but now within the for loop over groups}
\CommentTok{# you add the calculation over observations (i.e. movies)}
\CommentTok{# substract the group mean from each observation, and sum these}
\CommentTok{# add these across groups to get the within variance}
\NormalTok{dff<-}\StringTok{ }\KeywordTok{na.omit}\NormalTok{(df[,}\KeywordTok{c}\NormalTok{(}\StringTok{"Opening_revenues"}\NormalTok{, }\StringTok{"MPAA_rating"}\NormalTok{)])}
\NormalTok{y<-}\StringTok{ }\NormalTok{dff}\OperatorTok{$}\NormalTok{Opening_revenues}
\NormalTok{x<-}\StringTok{ }\NormalTok{dff}\OperatorTok{$}\NormalTok{MPAA_rating}
\NormalTok{groups <-}\StringTok{ }\KeywordTok{levels}\NormalTok{(}\KeywordTok{factor}\NormalTok{(dff}\OperatorTok{$}\NormalTok{MPAA_rating))}

\NormalTok{within <-}\StringTok{ }\DecValTok{0}
\ControlFlowTok{for}\NormalTok{ (group }\ControlFlowTok{in}\NormalTok{ groups) \{}
\NormalTok{  nj <-}\StringTok{ }\KeywordTok{length}\NormalTok{(y[x}\OperatorTok{==}\NormalTok{group])}
\NormalTok{  w<-}\StringTok{ }\NormalTok{(y[x}\OperatorTok{==}\NormalTok{group]}\OperatorTok{-}\KeywordTok{mean}\NormalTok{(y[x}\OperatorTok{==}\NormalTok{group]))}\OperatorTok{^}\DecValTok{2}
\NormalTok{  within <-}\StringTok{ }\NormalTok{within}\OperatorTok{+}\KeywordTok{sum}\NormalTok{(w)}
\NormalTok{\}}
\NormalTok{within}\OperatorTok{/}\KeywordTok{length}\NormalTok{(y)}\OperatorTok{-}\KeywordTok{length}\NormalTok{(groups)}
\end{Highlighting}
\end{Shaded}

\begin{verbatim}
## [1] 5.602047e+14
\end{verbatim}

\hypertarget{reference-category}{%
\section{5. Reference category}\label{reference-category}}

Re-estimate your regression under 2 by altering the reference category
to the R rated movies. Use the \texttt{relevel} and report the
\(F\)-value. How would you interpret the intercept now?

\begin{Shaded}
\begin{Highlighting}[]
\CommentTok{# write your R code here}
\NormalTok{lm5 <-}\StringTok{ }\KeywordTok{lm}\NormalTok{(df}\OperatorTok{$}\NormalTok{Opening_revenues }\OperatorTok{~}\StringTok{ }\KeywordTok{relevel}\NormalTok{(}\KeywordTok{factor}\NormalTok{(df}\OperatorTok{$}\NormalTok{MPAA_rating), }\StringTok{"R"}\NormalTok{))}
\KeywordTok{summary}\NormalTok{(lm5)}
\end{Highlighting}
\end{Shaded}

\begin{verbatim}
## 
## Call:
## lm(formula = df$Opening_revenues ~ relevel(factor(df$MPAA_rating), 
##     "R"))
## 
## Residuals:
##       Min        1Q    Median        3Q       Max 
## -18056954 -12850789  -8871258   4565804 189364184 
## 
## Coefficients:
##                                           Estimate Std. Error t value Pr(>|t|)
## (Intercept)                               10207655    1221406   8.357  < 2e-16
## relevel(factor(df$MPAA_rating), "R")G     -1273879    6454826  -0.197    0.844
## relevel(factor(df$MPAA_rating), "R")PG     2816319    2391770   1.178    0.239
## relevel(factor(df$MPAA_rating), "R")PG-13  7866870    1623975   4.844 1.47e-06
##                                              
## (Intercept)                               ***
## relevel(factor(df$MPAA_rating), "R")G        
## relevel(factor(df$MPAA_rating), "R")PG       
## relevel(factor(df$MPAA_rating), "R")PG-13 ***
## ---
## Signif. codes:  0 '***' 0.001 '**' 0.01 '*' 0.05 '.' 0.1 ' ' 1
## 
## Residual standard error: 23720000 on 1011 degrees of freedom
##   (138 observations deleted due to missingness)
## Multiple R-squared:  0.0239, Adjusted R-squared:  0.02101 
## F-statistic: 8.253 on 3 and 1011 DF,  p-value: 1.991e-05
\end{verbatim}

\begin{Shaded}
\begin{Highlighting}[]
\CommentTok{## F-value: 8.253}
\end{Highlighting}
\end{Shaded}

Why did the beta coefficient of PG-13 change wrt 2?

\hypertarget{multiple-regression}{%
\section{6. Multiple regression}\label{multiple-regression}}

Regress IMDB rating and country on opening revenues. Before you start,
create a complete dataframe (with only the three variables needed) using
\texttt{na.omit}, and keep this in a separate object. \textbf{This step
requires standardizing. To make sure R performs standardization on the
complete dataset, we will first create a new dataframe of complete
cases.}. Perform the analysis on this new complete dataframe. Save the
\texttt{lm} object in R. Report standardized coefficients. Use
\texttt{scale} on all \emph{numeric} variables.

\begin{Shaded}
\begin{Highlighting}[]
\CommentTok{# write your R code here}
\NormalTok{df6 <-}\StringTok{ }\KeywordTok{na.omit}\NormalTok{(df[,}\KeywordTok{c}\NormalTok{(}\StringTok{"imdb.com_rating"}\NormalTok{, }\StringTok{"Country"}\NormalTok{, }\StringTok{"Opening_revenues"}\NormalTok{)])}
\NormalTok{lm6 <-}\StringTok{ }\KeywordTok{lm}\NormalTok{(}\KeywordTok{scale}\NormalTok{(df6}\OperatorTok{$}\NormalTok{Opening_revenues) }\OperatorTok{~}\StringTok{ }\KeywordTok{scale}\NormalTok{(df6}\OperatorTok{$}\NormalTok{imdb.com_rating) }\OperatorTok{+}\StringTok{ }\NormalTok{df6}\OperatorTok{$}\NormalTok{Country)}
\KeywordTok{summary}\NormalTok{(lm6)}
\end{Highlighting}
\end{Shaded}

\begin{verbatim}
## 
## Call:
## lm(formula = scale(df6$Opening_revenues) ~ scale(df6$imdb.com_rating) + 
##     df6$Country)
## 
## Residuals:
##     Min      1Q  Median      3Q     Max 
## -1.3072 -0.4028 -0.0594  0.0839  7.4556 
## 
## Coefficients:
##                            Estimate Std. Error t value Pr(>|t|)    
## (Intercept)                -0.58513    0.04055 -14.429  < 2e-16 ***
## scale(df6$imdb.com_rating)  0.12007    0.02662   4.511  7.2e-06 ***
## df6$CountryUSA              1.02720    0.05374  19.115  < 2e-16 ***
## ---
## Signif. codes:  0 '***' 0.001 '**' 0.01 '*' 0.05 '.' 0.1 ' ' 1
## 
## Residual standard error: 0.8552 on 1031 degrees of freedom
## Multiple R-squared:   0.27,  Adjusted R-squared:  0.2686 
## F-statistic: 190.6 on 2 and 1031 DF,  p-value: < 2.2e-16
\end{verbatim}

Compare with the predicted values of the regression. Use
\texttt{predict} to get the predicted scores from the \texttt{lm} object
in R. Explore the mean predicted value for each country using
\texttt{tapply}.

\begin{Shaded}
\begin{Highlighting}[]
\CommentTok{# write your R code here}
\KeywordTok{tapply}\NormalTok{(}\KeywordTok{predict}\NormalTok{(lm6), df6}\OperatorTok{$}\NormalTok{Country, mean, }\DataTypeTok{na.rm =}\NormalTok{ T)}
\end{Highlighting}
\end{Shaded}

\begin{verbatim}
## Netherlands         USA 
##  -0.5813188   0.4391967
\end{verbatim}

Inspect the regression assumptions of the multiple regression. Determine
whether the regression assumptions are violated.

\begin{Shaded}
\begin{Highlighting}[]
\KeywordTok{par}\NormalTok{(}\DataTypeTok{mfrow=}\KeywordTok{c}\NormalTok{(}\DecValTok{2}\NormalTok{,}\DecValTok{2}\NormalTok{)) }\CommentTok{# here I make sure to plot 4 graphs next and below to another}
\CommentTok{# write your R code here}
\KeywordTok{plot}\NormalTok{(lm6)}
\end{Highlighting}
\end{Shaded}

\includegraphics{BRM1-week4_files/figure-latex/unnamed-chunk-12-1.pdf}

\begin{Shaded}
\begin{Highlighting}[]
\KeywordTok{dev.off}\NormalTok{()}
\end{Highlighting}
\end{Shaded}

\begin{verbatim}
## null device 
##           1
\end{verbatim}

As you can observe, the residuals are not normally distributed, and
violates the assumption of normality. In the following exercises a
transformation is used in an attempt to make the residuals more normally
distributed.

\hypertarget{log-transformation}{%
\section{7. Log transformation}\label{log-transformation}}

Rerun the model you've estimated in exercise 6 but now use a logarithmic
scale for the dependent variable with \texttt{log()} and do not
standardize the numeric variable(s). Calculate the predicted value for a
movie in country 1 for an IMDB-rating of 8.5. Make sure predicted value
is measured at the original scale using \texttt{exp()}.

\begin{Shaded}
\begin{Highlighting}[]
\CommentTok{# write your R code here}
\NormalTok{lm7 <-}\StringTok{ }\NormalTok{(}\KeywordTok{lm}\NormalTok{(}\KeywordTok{log}\NormalTok{(df}\OperatorTok{$}\NormalTok{Opening_revenues) }\OperatorTok{~}\StringTok{ }\NormalTok{df}\OperatorTok{$}\NormalTok{imdb.com_rating }\OperatorTok{+}\StringTok{ }\KeywordTok{factor}\NormalTok{(df}\OperatorTok{$}\NormalTok{Country)))}
\NormalTok{lm7}\OperatorTok{$}\NormalTok{coefficients[}\DecValTok{1}\NormalTok{] }\OperatorTok{+}\StringTok{ }\NormalTok{lm7}\OperatorTok{$}\NormalTok{coefficients[}\DecValTok{2}\NormalTok{]}\OperatorTok{*}\FloatTok{8.5} \OperatorTok{+}\StringTok{ }\NormalTok{lm7}\OperatorTok{$}\NormalTok{coefficients[}\DecValTok{3}\NormalTok{]}\OperatorTok{*}\DecValTok{0}
\end{Highlighting}
\end{Shaded}

\begin{verbatim}
## (Intercept) 
##    12.62435
\end{verbatim}

\begin{Shaded}
\begin{Highlighting}[]
\KeywordTok{exp}\NormalTok{(lm7}\OperatorTok{$}\NormalTok{coefficients[}\DecValTok{1}\NormalTok{] }\OperatorTok{+}\StringTok{ }\NormalTok{lm7}\OperatorTok{$}\NormalTok{coefficients[}\DecValTok{2}\NormalTok{]}\OperatorTok{*}\FloatTok{8.5} \OperatorTok{+}\StringTok{ }\NormalTok{lm7}\OperatorTok{$}\NormalTok{coefficients[}\DecValTok{3}\NormalTok{]}\OperatorTok{*}\DecValTok{0}\NormalTok{)}
\end{Highlighting}
\end{Shaded}

\begin{verbatim}
## (Intercept) 
##    303868.9
\end{verbatim}

\begin{Shaded}
\begin{Highlighting}[]
\KeywordTok{summary}\NormalTok{(lm7)}
\end{Highlighting}
\end{Shaded}

\begin{verbatim}
## 
## Call:
## lm(formula = log(df$Opening_revenues) ~ df$imdb.com_rating + 
##     factor(df$Country))
## 
## Residuals:
##     Min      1Q  Median      3Q     Max 
## -5.1162 -0.5153  0.0794  0.6712  2.6016 
## 
## Coefficients:
##                       Estimate Std. Error t value Pr(>|t|)    
## (Intercept)           12.36396    0.23304  53.054   <2e-16 ***
## df$imdb.com_rating     0.03063    0.03542   0.865    0.387    
## factor(df$Country)USA  3.93663    0.07104  55.416   <2e-16 ***
## ---
## Signif. codes:  0 '***' 0.001 '**' 0.01 '*' 0.05 '.' 0.1 ' ' 1
## 
## Residual standard error: 1.131 on 1031 degrees of freedom
##   (119 observations deleted due to missingness)
## Multiple R-squared:  0.7487, Adjusted R-squared:  0.7482 
## F-statistic:  1536 on 2 and 1031 DF,  p-value: < 2.2e-16
\end{verbatim}

\hypertarget{predict-with-and-without-transformation}{%
\section{8. Predict with and without
transformation}\label{predict-with-and-without-transformation}}

Rerun the unstandardized regression of exercise 7 but now without the
log transformation. Calculate the predicted value for a movie in country
1 for an IMDB-rating of 8.5. Compare with the predicted value based on
model estimated in exercise 7. Can you explain the difference?

\begin{Shaded}
\begin{Highlighting}[]
\CommentTok{# write your R code here}
\NormalTok{lm8 <-}\StringTok{ }\NormalTok{(}\KeywordTok{lm}\NormalTok{((df}\OperatorTok{$}\NormalTok{Opening_revenues) }\OperatorTok{~}\StringTok{ }\NormalTok{df}\OperatorTok{$}\NormalTok{imdb.com_rating }\OperatorTok{+}\StringTok{ }\KeywordTok{factor}\NormalTok{(df}\OperatorTok{$}\NormalTok{Country)))}
\NormalTok{lm8}\OperatorTok{$}\NormalTok{coefficients[}\DecValTok{1}\NormalTok{]}\OperatorTok{+}\NormalTok{lm8}\OperatorTok{$}\NormalTok{coefficients[}\DecValTok{2}\NormalTok{]}\OperatorTok{*}\FloatTok{8.5}\OperatorTok{+}\NormalTok{lm8}\OperatorTok{$}\NormalTok{coefficients[}\DecValTok{3}\NormalTok{]}\OperatorTok{*}\DecValTok{0}
\end{Highlighting}
\end{Shaded}

\begin{verbatim}
## (Intercept) 
##     6486475
\end{verbatim}

\begin{Shaded}
\begin{Highlighting}[]
\KeywordTok{summary}\NormalTok{(lm8)}
\end{Highlighting}
\end{Shaded}

\begin{verbatim}
## 
## Call:
## lm(formula = (df$Opening_revenues) ~ df$imdb.com_rating + factor(df$Country))
## 
## Residuals:
##       Min        1Q    Median        3Q       Max 
## -31144266  -9596378  -1415127   1998753 177630626 
## 
## Coefficients:
##                        Estimate Std. Error t value Pr(>|t|)    
## (Intercept)           -17988273    4200152  -4.283 2.02e-05 ***
## df$imdb.com_rating      2879382     638347   4.511 7.20e-06 ***
## factor(df$Country)USA  24473360    1280318  19.115  < 2e-16 ***
## ---
## Signif. codes:  0 '***' 0.001 '**' 0.01 '*' 0.05 '.' 0.1 ' ' 1
## 
## Residual standard error: 20380000 on 1031 degrees of freedom
##   (119 observations deleted due to missingness)
## Multiple R-squared:   0.27,  Adjusted R-squared:  0.2686 
## F-statistic: 190.6 on 2 and 1031 DF,  p-value: < 2.2e-16
\end{verbatim}

\end{document}
